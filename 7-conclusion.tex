\chapter{Conclusion and Recommendations}
\label{ch:conclusion}
The aim of this report was to evaluate the possibilities and capabilities for surveying Near-Earth Asteroids using a system of multiple spacecraft. The main question to be answered was: ``What is the optimal position and composition for a system of spacecraft with the purpose of identifying and cataloguing previously unidentified Near-Earth Asteroids?''. To conclude the report, here the main question and associated subquestions will be answered, and conclusions will be drawn from these answers which can be used in the design of future NEA survey missions. In addition, during research several opportunities were found for further research to better understand the capabilities of multi-spacecraft NEA surveys and to further develop the technology required to realise them. These opportunities will be turned into recommendations for future research.

\section{Conclusions}
Before summarizing the conclusions of the report, concrete answers to the research questions drafted in \autoref{sec:researchquestions} will be provided. Firstly, the various subquestions can be adressed:

\begin{enumerate}
 \item \textbf{How can the population of NEAs be accurately modelled, and how can these models be adjusted for unidentified NEA's?}: Modelling the population of NEA's was performed using a model created with the aid of NEOWISE data, compensated for biases in observation. This model was published by \cite{GranvikPopulation}, and is publicly available. As research focussed on still unidentified NEA's, the population was corrected with completeness statistics calculated from NEOWISE data by \cite{HarrisPopulation}. 
 \item \textbf{How can surveys of NEA's by a system of spacecraft be accurately modelled?}: Surveys were modelled by explicit calculation of the important steps in the process. First, the positions of all asteroids and spacecraft are calculated according to Keplerian orbits. Then, the background signal and target signals are calculated. From this, using representative hardware parameters, the signal-to-noise ratio can be obtained. Lastly, a probabilistic detection model is used to establish detection, and integration of the system in time allows for establishing identification. Components of the system were found from various sources in peer-reviewed literature, and their implementation was thoroughly verified and validated. In addition, it was shown that the simulation yielded similar results as other, previously validated survey models.
 \item \textbf{How can the position and composition of the system be optimized?}: The main challenge involved in optimizing the system is the fact that no useful analytical properties, such as gradients, are available, and the fact that the survey model is computationally expensive. For this reason, an approach of surrogate optimization was implemented. It was shown that this optimization yielded results close to what was expected to be the optimum. However, when larger numbers of parameters were introduced into the optimization, overfitting reduced the accuracy of the results.
 \item \textbf{What is the effect of increasing the number of spacecraft on the process and performance of identifying and cataloguing NEA's?}: It was found that performance continuously increases with an increasing number of spacecraft. Initially, a second spacecraft yields very high improvements in performance - close to 50\% for thermal infrared systems - due to the possibility of performing triangulation, allowing for faster NEA identification. As the number of spacecraft further increases, the relative performance increase exponentially decreases. Therefore, systems with a large number of spacecraft will most likely not be cost-efficient compared to systems with less spacecraft, however addition of a second or third spacecraft will yield sizeable performance increases nonetheless.
 \item \textbf{How is the performance of possible payload compositions affected by the number of spacecraft, and what is the resulting optimal payload composition?}: It was shown that thermal infrared telescope payloads are the best choice for a future deep space NEA survey mission. Not only are thermal infrared telescopes superior to visual light telescopes for single spacecraft systems, the research presented in this report also shows that the relative performance increase of a thermal infrared system as the number of spacecraft is increased is higher than that of a visual light system. In addition, it was shown that for practical numbers of spacecraft, so-called ``hybrid'' systems with both visual light and thermal infrared telescopes did not provide enough synergistic benefits to outweight the worse performance of the visual light telescope.
 \item \textbf{How do the number of spacecraft and payload interact with the orbital parameters of the system?}: It was found that optimally, the spacecraft should be situated in a circular orbit, equally spaced out. The semi-major axis of said orbit increases as the number of spacecraft increases, because the spacecraft are placed closer together, introducing an inefficient overlap in their observation areas.
 \item \textbf{How effective is a system of multiple spacecraft at identifying and cataloguing previously unidentified NEA's compared to other current and future methods?}: A system of multiple spacecraft was shown to provide significant performance benefits to a single spacecraft system. Addition of a second spacecraft is expected to increase the improvement in completeness granted by a single spacecraft system by more than 50\% for asteroids in the tens to hundreds of meters in diameter. Further increases in the number of spacecraft will still yield improvements, however as the number of spacecraft increases, the majority of improvement will occur in the small, sub-100m diameter ranges.
\end{enumerate}

After answering the subquestions, the main research question can be adressed: ``What is the optimal position and composition for a system of spacecraft with the purpose of identifying and cataloguing previously unidentified Near-Earth Asteroids?'' Through the research, it was found that the optimal NEA survey using multiple spacecraft should utilize the following:
\begin{itemize}
 \item \textbf{Number of spacecraft}: The number of spacecraft is dependent on the desired performance. It was shown that increasing the number of spacecraft will continue to increase the performance of the system, even when the system contains tens to hundreds of spacecraft. However, diminishing returns quickly set it and economic considerations should be included if an optimum is to be found.
 \item \textbf{Orbit}: The optimal orbit for the system was found to be a circular orbit, with the spacecraft spread out over the full circle of the orbit. In this way, the total volume of space effectively covered by the spacecraft is maximized, and does not vary in time.
 \item \textbf{Orbital radius}: The semi-major axis of the orbit should increase with increasing number of spacecraft, to reduce the overlap in spacecraft observation coverage as the distance between spacecraft reduces. Initially, for a single spacecraft, the optimum is found around 0.8 AU. For 2 or 3 spacecraft systems, the optimum lies around 1.0 AU. For a 4 to 5 spacecraft system, the optimal radius increases to around 1.1 AU. Further spacecraft will yield further increase of the semi-major axis. It was also found that the semi-major axis is not a very sensitive parameter for the performance, and deviation of up to 0.1 AU from the optimum will result in no statistically significant degradation of performance.
 \item \textbf{Payload}: Lastly, the system should be composed of spacecraft using telescopes built for imaging in the thermal infrared spectrum. As already shown by previous research, thermal infrared is preferred over visual light systems, as the background signal is weaker, leading to better signal-to-noise ratios for small targets. In addition, this research has shown that the relative benefit of using multiple spacecraft is greater when using thermal infrared telescopes. In addition, it is expected that thermal infrared sensor technology will still improve substantially in the coming years, adressing current shortcomings such as small sensor field-of-views and long integration times. This will improve the performance even further.
\end{itemize}

In conclusion, an analysis was provided of the behavior, ideal parameters, and expected performance of a multi-spacecraft NEA survey. Through answering the research question, a set of guidelines was obtained which can be used either directly in a design effort towards further surveying missions, or as a basis for future research. To facilitate in carrying out this research, the next section will provide some recommendations with regards to possible avenues to explore next.

\section{Recommendations for Further Research}
Throughout the research, several opportunities were identified to improve the understanding of multi-spacecraft NEA systems, or to further develop the necessary technologies. These form the basis of the recommendations listed below:
\begin{itemize}
 \item \textbf{NEA survey search strategies}: Currently, very little literature exists on the strategy that should be used for searching for NEA's. Although algorithms exists for single-spacecraft surveys, they are not extensively researched. In addition, no work has been done to exploit the possibilities of multi-spacecraft search strategies. For example, telescopes could focus their search effort in an area where another telescope has detected an NEA, or search strategies could be set up to maximize the occurence of triangulation, speeding up detection efforts significantly and helping to detect small, fast travelling NEA's.
 \item \textbf{On-board processing and communication limitations}: Throughout this report, it was assumed that spacecraft are capable of processing images from the telescopes and communicating within the system and to Earth. However, neither of these capabilities are currently proven (see \cite{LiteratureReview} for a full exploration). Therefore, work should be done on how a multi-spacecraft system can efficiently process and communicate the survey data.
 \item \textbf{Adressing the overfitting of the optimizer}: Detailed optimization efforts were hindered by a growing amount of overfit manifesting in the optimization efforts. As this thesis is not focussed on the topic of machine learning, and sufficient conclusions could be drawn, these results were left at that. However, to confirm the expectation that no significantly better solutions exists, it is suggested to attempt an optimization method which is less susceptible to noise in the signal. The recommendation would be to first attempt a different surrogate model, such as using an artificial neural network, instead of a classical machine learning approach, and possibly branching our further from that.
 \item \textbf{Investigation of the problem in non-Keplerian celestial mechanics}: Although the error in the solution is expected to be small, it is recommended to validate this by performing a three-body or n-body simulation of the problem. This might also interplay with advances in the search strategies, as more subtleties in the orbits of NEA's become important for the performance. In addition, non-Keplerian mechanics might allow exploitation of different orbital configurations for the system of spacecraft. In particular, utilization of Lagrange point could allow for implementation of differing semi-major axes per spacecraft, whilst still maintaining a constant distance between the spacecraft throughout their mission.
\end{itemize}
