\listoffigures
\listoftables

\chapter*{List of Symbols}

\begin{table}[h!]
\centering
\begin{tabular}{l|l|l}
\textbf{Symbol} & \textbf{Meaning}                         & \textbf{Unit} \\ \hline
$a$               & semi-major axis                          & AU            \\
$A$               & bond albedo                              & -             \\
$A$               & aperture                                 & m$^2$            \\
$A_H$              & area covered at magnitude H              & AU$^2$           \\
$A_{total}$          & total area                               & AU$^2$           \\
$b_e$              & heliocentric ecliptic latitude           & deg           \\
$b_g$              & galactic latitude                        & deg           \\
$C$               & completeness                             & \%            \\
$D$               & diameter                                 & m             \\
$D$               & dark current                             & e$^-$s$^{-1}$          \\
$e$               & eccentricity                             & -             \\
$E_c$              & emissivity of component c                & -             \\
$F$               & flux                                     & Wm$^{-2}$          \\
$H$               & absolute magnitude                       & -             \\
$i$               & inclination                              & deg           \\
$k_f$              & straddle factor                          & -             \\
$l_e$              & heliocentric ecliptic longitude          & deg           \\
$l_g$              & galactic longitude                       & deg           \\
$n$               & number                                   & -             \\
$n_c$            & density of component c                   & 1/AU  \\
$P$               & probability                              & \%            \\
$p_v$              & assumed albedo                           & -             \\
$Q_e$              & quantum efficiency                       & \%            \\
$Q$              & apohelion distance                       & AU            \\
$q$              & perihelion distance                       & AU            \\
$R$               & heliocentric distance of spacecraft                   & AU            \\
$r$               & heliocentric distance of target          & AU            \\
$R$               & readout noise                            & e$^-$            \\
$S_b$              & Background signal                        & e$^-$            \\
$S_t$              & target signal                            & e$^-$            \\
$T$               & temperature                              & K             \\
$T$               & period                                   & s$^{-1}$           \\
$t$               & time                                     & s             \\
$V$               & apparent magnitude                       & -             \\
$Z$               & zodiacal flux                            & Wm$^{-2}$          \\
$\alpha$           & solar phase angle                        & deg           \\
$\Delta$           & distance from target to observer         & AU            \\
$\eta$             & beaming parameter                        & -             \\
$\phi$             & angular distance from the subsolar point & deg           \\
$\phi_{max}$         & maximum solar elongation                 & deg           \\
$\lambda$          & wavelength                               & m             \\
$\tau$             & integration time                         & s             \\
$\Theta$           & field of view                            & deg$^2$          \\
$\theta$           & mean anomaly at epoch                    & deg           \\
$\Delta \theta$     & inter-spacecraft spread                  & deg           \\
$\omega$           & argument of periapsis                    & deg           \\
$\Omega$           & right ascension of the ascending node    & deg          
\end{tabular}
\end{table}

\chapter*{List of Abbreviations}

\begin{table}[h!]
\centering
\begin{tabular}{l|l}
\textbf{Abbrevations} & \textbf{Meaning}                                                \\ \hline
ANN                   & artificial neural network                                       \\
ATLAS                 & Asteroid Terrestrial-Impact Last Alert System                   \\
$b$                     & body-centered, ecliptic reference frame                         \\
CCD                   & charge-coupled device                                           \\
COBE                  & Cosmic Background Explorer                                      \\
CSS                   & Catalina Sky Survey                                             \\
CURR                  & current                                                         \\
$e$                     & heliocentric, ecliptic reference frame                          \\
$g$                     & galactic reference frame                                        \\
$h$                     & body-centered, ecliptic reference frame oriented w.r.t. the Sun \\
JPL                   & Jet Propulsion Laboratory                                       \\
$L_1$, $L_2$, $L_3$            & Lagrange points                                                 \\
LSST                  & Large Synoptic Survey Telescope                                 \\
NASA                  & National Aeronautics and Space Administration                   \\
NEA                   & near-Earth asteroid                                             \\
NEATM                 & near-Earth asteroid thermal model                               \\
NEOCam                & Near-Earth Object Camera                                        \\
NEOWISE               & Near-Earth Object WISE                                          \\
PROJ                  & projection                                                      \\
S/C                   & spacecraft                                                      \\
SNR                   & signal-to-noise ratio                                           \\
TIR                   & thermal infrared                                                \\
VIS                   & visual light                                                    \\
WISE                  & Widefield Infrared Survey Explorer                              \\
$\odot$                  & the Sun                                                         \\
$\oplus$                 & the Earth                                                      
\end{tabular}
\end{table}

\chapter*{List of Constants}

\begin{table}[h!]
\centering
\begin{tabular}{l|l|l|l}
\textbf{Constant} & \textbf{Meaning}                    & \textbf{Value} & \textbf{Unit} \\ \hline
$h$                 & Planck constant                     & 6.62607015     & JHz$^{-1}$         \\
$c$                 & speed of light                      & 299,792,458      & ms$^{-1}$           \\
$\sigma$             & Stefan-Boltzmann constant           & 5.67$\cdot 10^{-8}$       & Wm$^{-2}$K$^{-4}$       \\
$k$                 & Palomar-Leiden exponent             & 2.95 to 3.5    & -             \\
$b_{NGP}$              & laittude of the North Galactic Pole & 29.81          & deg           \\
$l_{NGP}$              & ascending node of the Milky Way     & 270.02         & deg           \\
$l_{GC}$               & longitude of the Galactic Core      & 6.38           & deg           \\
$F_{\odot}$            & Solar flux at 1 AU                  & 1361           & W/m$^2$         
\end{tabular}
\end{table}
