\chapter{Research Outline}
In \autoref{ch:introduction}, some background on the difficulties of identification and cataloguing of NEAs was given. In this chapter, the resulting problem and associated knowledge gap will be presented. Then, the associated research questions and expected outcomes will be listed.

\section{Problem Statement}
Currently, humanity's knowledge of NEA populations is at a low level of completeness, especially for small diameter NEAs. Therefore, valuable scientific knowledge about the composition and evolution of the Solar system is unknown, and Earth is vulnerable to impacts which can be hazardous to human life and property. \\

It has been shown that current efforts are not adequate to reach the current goal of the spaceguard survey. Several missions have been proposed, and others are under development, which will cover this goal. However, a new more ambitious goal to identify smaller NEAs is still far out of reach. Even with a modern satellite positioned in deep space, only a limited survey completeness can be reached at limiting diameters $D < 100 \mathrm{m}$. This is caused by the limitations in position of this system, the required follow-up time and the number of detections required, and interference from the Sun.

\section{A Multi-Spacecraft Approach}
\label{sec:researchmultispacecraft}
To address this problem, we propose the option of a multi-spacecraft system. In recent years, spacecraft constellations have already shown a lot of potential in reaching complex mission goals. In the application of near-Earth asteroid surveys, more telescopes will firstly speed up the survey cadence, allowing the system to image the same area of sky at a faster rate. However, there are further synergystic advantages to such an approach. Three major benefits are noted in a multi-spacecraft system over a single telescope, which will be discussed in the following paragraphs.\\

Firstly, a multi-spacecraft system will mostly solve the problem of Solar glare: Although a spacecraft in syzygy with the Sun and an asteroid will not be able to detect the latter if it is located in the direction of the Sun, a different spacecraft located away from it might observe the Sun-asteroid arrangement from the side, allowing it to detect the target. In this way, a multi-spacecraft system is capable of minimizing the amount of blind spots in the search space. \\

Secondly, using multiple spacecraft allows for easier identification and orbit determination of the NEA: Normally, a single telescope takes images in 2D of a target. As the asteroid will almost certainly be below the Rayleigh criterion of the telescope, it is not possible to estimate how close the asteroid is from its estimated diameter and the projected size on the sensor. Therefore, only the angular direction towards the target is known. Therefore, to obtain the orbit of the target requires solving Gauss' problem, which requires a minimum of three subsequent observations (six unknown parameters for the full orbit specification, two variables measured per observation). When using multiple spacecraft, it is possible to perform a kind of ``triangulation'', provided the spacecraft and the asteroid are not colinear. This allows for solving for the three-dimensional position of the asteroid. Thus, using only two observations in time reduces the orbit determination to Lambert's problem. This means the asteroid will only have to be within the area where telescopes can observe it for half the time as a single-spacecraft system. \\

Lastly, a multi-spacecraft approach allows for more complex search strategies. The possibility for doing such search strategies when multiple sensors are available is demonstrated by the Cataline Sky Survey. Using their follow-up telescope, a new target is quickly selected for follow-up imaging, quickly gathering the required observations to perform orbit determination and thereby identification. In space, such a strategy would of course by more complex, as the problem becomes influenced by the location of the spacecraft. However, such an implementation will be very helpful in detecting NEAs which are only visible for a short period of time, such as highly eccentric objects with long semi-major axis, which are only visible for a short window around their perihelion. \\

However, before such strategies can be developed and implemented in a mission, it is first required to know the location and composition of such a multi-spacecraft system. To the author's best knowledge, the behavior of a multi-spacecraft survey has never been evaluated. In addition, to adequately assess the performance, it is vital to discover where the craft in such a system should be positioned, and how they should be equipped. The aim of this work are to provide insight into these points using simulation of such a survey.

\section{Research Questions and Expected Outcomes}
\label{sec:researchquestions}
