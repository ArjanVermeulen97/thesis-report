\chapter*{Preface}
\setheader{Preface}

Starting out, 7 years ago, the entire idea of writing a thesis and graduating seemed foreign, and far away. How is it possible to take such a vague assignment and develop it in a meaningful way? Yet here we are, and for some reason these things always seem to go faster than anticipated. It's been a long, \textit{interesting} journey here in Delft, but I wouldn't have wanted to miss it in any way. Next to meeting and interacting with a diverse group of people, the general skills and development in \textit{thinking} about things will prove for ever valuable: looking back at previous projects, my own skills not just in designing flying things, but more general in analysing and thinking critically about problems, have progressed tremendously. I wonder how it will be like looking back on this document in a few years time...\\

Next to all the people in my life who've had to deal with my busy schedules and stressed weekend nights, and inspiring and motivating me to continue working on it: parents, family, friends and colleagues, I'm also especially grateful for the wonderful people of the faculty of AE: From the students, to the support staff, teachers, and other academic personell, they've made a lasting impression and truly made the experience into what it is. \\

Among those people, let's not forget a special mention for the two people of the faculty I've been working with the past year: Jian Guo and Jeannette Heiligers, my two supervisors. Although initially starting out the project in an entirely different direction, we've ended up at a fascinating blend of both sides of the spaceflight coin. From guiding me in the right direction to actually obtain a useful research question and plan at the beginning of my work, all the way to challenging sometimes the most minute statements in my final conclusions, I could not have delivered this work without their input, questions, and discussion. Sometimes indeed the correct approach is not to dive straight back into work, but to take a step back and actually \textit{think}. Often, that thinking proves that the simplest, most straightforward comments provide the largest headaches.\\

Dear reader, I would lastly like to thank \textit{you}, for taking the time to read this report, and I hope you'll enjoy reading it as much as I enjoyed making it.

\begin{flushright}
{\makeatletter\itshape
    \@author \\
    Delft, February 2022
\makeatother}
\end{flushright}

