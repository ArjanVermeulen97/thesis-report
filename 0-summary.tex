\chapter{Summary}

Asteroids are a well-known threat to life on Earth. Since discovering that a large asteroid was responsible for the extinction of the non-avian dinosaurs, 66 million years ago, humanity has been aware of the threat of impacts from space. Through numerous surveys over the past decades, humanity has achieved a substantial amount of knowledge about the asteroid population. The goal which is currently persued - cataloguing 90\% of all asteroids over 140 meters in diameter - is expected to be completed in the upcoming years, and it is expected that all asteroids capable of global destruction have been identified - and known to not present any danger. However, knowledge of the population of asteroids below these sizes is very limited. Meanwhile, these smaller asteroids impact Earth far more frequently, and in some cases - such as the 2013 Chelyabinsk meteor - can damage numerous buildings and injure thousands of people. Next to defending humanity against asteroids, a complete overview of the Near-Earth Asteroid (NEA) population would increase our understanding of the evolution of the Solar system, and various interactions within it, to a deeper degree. Therefore, increasing the survey completeness of the NEA population is a topic worthy of scientific pursuit. \\

For these reasons, numerous efforts are currently underway to realize future NEA surveys. However, most surveys to date are conducted from Earth. While this allows easy access to electricity, computing infrastructure, and logistical support for very large telescopes, efforts are hindered by interference from day/night cycles, weather, and general distortion due to the atmosphere. Even the surveys that have been conducted from space, have been conducted from orbits near Earth. This still presents numerous problems, such as interference by Earth and Moon, and a different thermal environment. Therefore, currently numerous proposals are on the table for deep space NEA survey missions. Using telescopes far away from Earth, in more optimal locations, might increase our knowledge of the NEA population in a more effective way. In this report, initial research into a variation on these deep-space surveys is presented: carrying out these surveys using a system of multiple spacecraft. Not only does increasing the number of spacecraft increase the area of sky which can be imaged in any given time period, other synergistic effects are also present: Firstly, multiple spacecraft can cover eachothers blind spots, which are caused by interference of Solar glare or thermal limitations, thereby reducing the volume in which NEA's can not be detected. Secondly, a multi-spacecraft system allows for triangulation to determine the position of an asteroids, instead of only determining the direction to the asteroids as in a 2D image from a single telescope. This reduces the required number of observations to identify the asteroid and determine its orbit. Lastly, a multi-spacecraft system allows for implementing more complex search strategies where e.g. certain telescopes act as follow-up telescopes, attempting to quickly identify new findings of others spacecraft. \\

The research presented aims to answer the question: ``What is the optimal position and composition for a system of spacecraft with the purpose of identifying and cataloguing previously unidentified near-Earth asteroids?'' Therefore, the primary parameters of interest are the number of spacecraft in the system, their respective payloads - including whether a system which combines both payloads might work, and their orbits. To research this topic, a simulation was developed based on sources in literature. The aim of the simulation is to model the actual process of an asteroid survey completely. Therefore, first a debiased population model based on the current understanding of the population of NEA's is implemented. For each of these asteroids, the signal as it would be received by a spacecraft observing it is calculated. This is combined with knowledge of the background signal, and various hardware properties such as sensor noise, to calculate the signal-to-noise ratio. Based on this signal-to-noise ratio, a probabilistic detection model is implemented. If the system manages to achieve a sufficient number of detections within a reasonable time window, the asteroid is marked as ``identified''. The primary result at the end of a simulation is the survey completeness, the number of asteroids identified divided by the total number of asteroids in the population. \\

The survey completeness parameter was used as the main objective to optimize the problem. Initially, the various parameters of the problem were investigated using a grid-search method. Following this, a numerical optimization scheme based on the method of surrogate optimization was implemented. Here, a secondary function which is easier to optimize is fitted to the to-be-optimized function. This method was necessary as modelling a complete survey is computationally expensive, and therefore the number of required simulations should be limited. Initially, the parameters of the system were constrained to co-orbital solutions, where all spacecraft are in the same orbit, and only the anomaly at epoch differs. Later, several variations involving unique orbits per spacecraft were investigated to reach a definite conclusion. \\

Several conclusions were obtained with regards to the behavior and performance of the system, all of which form a basis for possible future design efforts or trade-offs. Firstly, it was determined that the performance of the system will continuously increase as the number of spacecraft increases. Initially, the increases compared to a single-spacecraft system are substantial, approximately 40\% for the addition of a second spacecraft, and a further 10\% for a third, fourth and fifth spacecraft each. However, diminishing returns quickly become an important factor, and further addition of spacecraft will yield only minor improvements, although these improvement continue to manifest even for a system of 100 spacecraft. The best solutions were found to comprise spacecraft in a single circular orbit, in such a way that the spacecraft are spread out as much as possible over the orbit. This maximizes the volume of space in which NEA's can be detected. Even when all spacecraft were allowed unique orbits, no solution was found which outperformed this circular solution in a significant way. The semi-major axis of the system should be dependent on the number of spacecraft, and the optimal point for it increases with increasing number of spacecraft, although there is a range of approximately $\pm$0.1 AU in which the system will operate near the optimum. \\

In the end, it was found that a two spacecraft system is capable of achieving approximately 60\% completion at 100 meter diameter, compared to only 45\% completion for a single spacecraft system. The difference is smaller at larger asteroid diameters, decreasing to an improvement from 85\% to 90\% at 500 meter diameter. Addition of further spacecraft will still yield an increase, although this increase is limited at NEA sizes over 100 meters. Primarily, larger numbers of spacecraft will improve the completeness at a small diameter. These result mean that such systems are a noteworthy consideration for future mission designs aiming to increase humanity's knowledge of the NEA population, to safeguard Earth from asteroid impact, or to conduct valueable scientific research.
