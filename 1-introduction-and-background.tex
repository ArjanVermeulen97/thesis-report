\chapter{Introduction}
\label{ch:introduction}
66 Million years ago, an asteroid the size of Rotterdam initiated what is perhaps the most well known cataclysmic event in the history of life on Earth. With an impact releasing the energy of a billion nuclear bombs, the asteroid left a 180 km crater in the Gulf of Mexico. Launching enough debris into the atmosphere to block out the light of the Sun, eventually leading to the extinction of three quarters of spiecies on Earth, most famously the non-avian dinosaurs (\cite{DinosaurAsteroid}). In recorded human history, a multitude of noteworthy asteroids have impacted Earth, such as the Tunguska impactor in 1908 in Siberia. Flattening over 2000 km$^2$ of forest, events such as this serve as a staunch reminder of the massive kinetic energy that can be released by an object descending to Earth from space, and the danger this poses to human civilization.\\

Cognizant of such hazard, the United States launched the Spaceguard Survey in 1992, aiming to ``identify 90\% of near-Earth Asteroids (NEA's) larger than 1 km within 10 years.'' (\cite{Spaceguard}). With improvements in observation technology, more meteors were witnessed and recorded, leading to greater awareness into the frequency and unpredictability of such events. Of course, impacts from space are not a problem exclusive to Earth; as the 1994 impact of comet Shoemaker-Levy 9 into Jupiter proved. This impact showed that impacts of objects large enough to cause global catastrophy were not as highly improbably as once considered, and asteroid identification efforts took off with it.\\

The initial spaceguard survey goal was completed succesfully, and it is known that there are - within reasonable probability - no civilization-ending asteroids destined for Earth impact in the coming millenium. Nevertheless, smaller asteroids can still pose a local threat to human life or property. In addition, much is still unknown about the exact population of near-Earth asteroids, and such knowledge might provide valuable insights into the origin and evolution of the Solar system. Therefore, NASA extended the spaceguard mandate to detect 90\% of all NEA's larger than 140m (\cite{SpaceguardHistory}). \\

Since then, a lot of progress has been made in cataloguing and identifying smaller NEA's. Additionally, consideration has been given to survey for smaller limiting diameters (e.g. \cite{2003NEOSDT}). However, such efforts have to date still been very unsuccesful: For example, in 2013, a meteoric airburst over the city of Chelyabinsk, Russia, seriously injured almost 1500 people and damaged several thousands of buildings. Although damage was limited due to the high altitude of the explosion, no precautionary measures were taken, as the asteroid was completely unknown until the moment of atmospheric entry. Luckily, such events are not a common occurence. However, the large majority of NEA's of this size is completely unknown, and as such they can strike anywhere at any time. \\

Largely, this lack of completeness can be attributed to the nature of current survey efforts: the majority of surveys are carried out from Earth, influenced by weather, daylight, and atmospheric interference. Even the surveys from space have always been conducted from orbits near Earth, subjecting them to unfavourable thermal environments and light from the Earth and Moon. Recently, several proposals have been made to undertake surveys from deep space. In this report, an extension to this idea is proposed for a multi-spacecraft system. After introducing the concept, and the various advantages of the system, research is performed to identify the optimal positions and compositions of such systems, as well as what performance to expect.
