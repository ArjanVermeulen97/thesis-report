\documentclass{article}
\usepackage[margin=3cm]{geometry}
\usepackage[utf8]{inputenc}

%opening
\title{Identification of Near-Earth Asteroids using Multi-Spacecraft Systems}
\author{J. G. P. Vermeulen, J. Guo, and M. J. Heiligers}
\date{}

\begin{document}

\maketitle

\noindent Through numerous survey efforts over the past decades, humanity has achieved substantial knowledge of the near-Earth asteroid (NEA) population. Nevertheless, survey completeness at very small asteroid sizes is still limited, and unannounced impacts, such as the 2013 Chelyabinsk meteor, are common enough to warrant further identification efforts. Because of the limitations of Earth-based surveys, several works have already investigated performing an NEA cataloguing survey from deep space using a single spacecraft. We propose an extension to this idea, where a multi-spacecraft system is utilized to perform such a survey. This offers several distinct advantages over a single spacecraft system, such as a decrease in blind spots due to solar interference, faster asteroid orbit determination through triangulation, and the possibility for more advanced search strategies.\\

\noindent A survey simulation tool was developed to approach the problem. Using a sample population of NEAs, the tool predicts the  expected survey completeness from the design parameters of the survey. Investigated parameters are the number of spacecraft; their payload, either visual light or thermal infrared telescopes; and the semi-major axis, eccentricity and mean anomaly of their orbits. For each timestep, the tool calculates the target and background signal in the relevant spectrum from each asteroid to each spacecraft. From these, the signal-to-noise ratio is determined which is used in a probabilistic detection model. Lastly, if sufficient detections are established in a 90-day period, the asteroid is labeled as identified.\\

\noindent In the first stage of the research, co-orbital configurations of spacecraft were studied. It was found that a circular circumsolar orbit with the spacecraft distributed evenly over the orbit provides the best results. Thermal infrared telescopes were determined to outperform visual light telescopes in all conditions. The optimal semi-major axis was found to increase with increasing number of spacecraft, starting at $0.9\mathrm{AU}$ for a single spacecraft, increasing by $0.03\mathrm{AU}$ per additional spacecraft.  The findings are supported by a novel hypothesis relating the expected survey completeness to the volume of space in which NEAs at varying limiting magnitudes can be effectively detected. Exploration of non-co-orbital arrangements yielded no results which significantly outperformed the co-orbital configurations, although it is noted that considerable problems were encountered in the optimization which preclude concluding that no better solution exists. Performance predictions indicate that a multi-spacecraft system of 2-3 spacecraft will identify 40-60\% more NEAs, relative to a single spacecraft, with strong diminishing returns occurring for higher numbers of spacecraft.

\end{document}
