\documentclass[whitelogo]{TUD-report2020}

\begin{document}
\chapter*{Identification of Near-Earth Asteroids using Multi-Spacecraft Systems}

For several decades, humanity has worked on cataloguing the population of near-Earth asteroids (NEA's). Knowledge of these small members of the Solar system not only helps in defending Earth from asteroid impacts; study of the population might also provide valuable scientific insights, and open economic possibilities through developments such as asteroid mining. Surveys from Earth currently face challenges in detecting the smallest NEA's: day- and nightcycles, weather patterns, and atmospheric distortion have encouraged several recent studies into the possibility of NEA surveys from deep space.\\

\noindent In this thesis, an extension to these proposals is studied: using multiple spacecraft co-operating on the survey. Next to the immediate increase in the data gathering capabilities, such systems offer synergistic benefits: Firstly, spacecraft can be placed in such a way as to cover eachothers blind spots caused by e.g. Solar glare. Secondly, imaging from multiple directions allows for triangulation to more quickly determine the orbit. Lastly, such a system could feature implementation of advanced search strategies, for example utilizing a part of the system as follow-up telescopes. \\

\noindent As one of the first works on predicting and optimizing the performance of such a multi-spacecraft NEA survey system, the work aims to provide a foundation on how to compose such a system, and what orbital configurations to select for its operation. A simulation tool was developed which explicitly models a NEA survey, and this model was validated against other research works and surveys. Using this tool, the behavior of the system is studied under changing of various parameters such as the number of spacecraft, thermal infrared or visual light telescopes, and various orbital elements. Following this, numerical optimization was performed to obtain conclusions with regards to the optimal composition and position of the system. The findings are supported by an investigation into the underlying principles driving the performance of the survey. Ultimately, results are obtained which can be used for more detailed studies into design of future NEA survey missions, or trade-offs against other concepts.
\end{document}
